\title{RW354 Project 1: Client-Server Chat Program}
\author{
        Murray Heymann \\
        15988694\\
                Department of Computer Science\\
        Stellenbosch University\\
}
\date{3 August 2016}

\documentclass[12pt]{article}

\begin{document}
\maketitle

\begin{abstract}
This is the report for the first project of RW354 in 2016. It will cover a
project description, a discussion of features implemented in the solution, 
a description of the data structures used in the implementation and finally
tests performed on the program.  
\end{abstract}

\section{Project Description}

\paragraph{}
We were required to write a chat program using the client-server model, to learn
more of practical computer networks and the problems faced in the implementation
of such a system. Any language could be used, but as java was recommended, java
NIO was used. Two applications were developed:  a server and a client.  The
server is run from a terminal, whereas the client can be run either using a
terminal, or with a graphical user interface.  

\paragraph{running}
Please see the README file on how to compile and run.  In short, the
shell scripts makeclient.sh, makeserver.sh, startclient.sh and startserver.sh
found in the src directory will do everything that is necessary to compile and run the project.

\section{Features Included}

\paragraph{Server}
The server runs two basic threads:  one for listening for incoming messages and
new incoming connections.  Java NIO allows for multiple connections to be
managed by a single thread and as starting up threads can become inefficient,
this was chosen as the preferred interface. Multiple clients can connect, as
long as each sends a unique username.  The other thread is responsible for
sending info to connected users. Basic actions are printed by the server
to the standard output, such as when a new user connects or disconnects and
communication being sent out to all users.  If another user tries to connect
with a username that is already online, he or she is refused.  When a user comes
online or goes offline, a list of users online are pushed to each user
remaining. The server should not crash,
regardless of whether the user closes the chat window, terminates the java
machine from the terminal or even force quits the terminal window.  It should
also be stable if the connection is broken.  The program was not tested for if
the client computer itself is force shutdown, as I love my computer and wasn't going to
risk it. In theory, the server should be able to handle that as well.

\paragraph{Client}
The client has a simple graphical user interface (GUI), although it can also be run in
"terminal" mode by providing this as an argument upon starting the client.  The
java swing code used for creating this graphical user interface is closely based
on code found on the site
"dream.in.code\footnote{http://www.dreamincode.net/forums/topic/259777-a-simple-chat-program-with-clientserver-gui-optional/}". 
Although the code was adapted, it very closely resembles code on this page and
credit is therefore given, as it is due.  
Upon startup, the GUI has two text areas where the server IP address and port
number can be entered, and two text areas for the name and password of the user.
Only the login button is active initially.  Logging in disables the login button
and activates the buttons for logging out, requesting a list of online users
(although this gets pushed by the server anyway), echoing a message back and
broadcasting a message to all users.  The text areas for the server IP address
and port number becomes uneditable.  A label now indicates that the other text
areas are now to be used for the name of the recipient and for entering the
message.  Buttons are provided for refreshing the list of users, for echoing a
message to oneself and for broadcasting to all users currently online.  If a
message entered is simply followed by an enter, a normal message is sent to the
recipient indicated as a private (whispered) message.  Messages are displayed in
a dedicated message area, with online users on a separate user area.

\section{Features Not Included}

\paragraph{Server}
Although the server does receive the password, and even has a method for
implementing password correctness, this method currently always always returns
true.  There is no database maintaining a lookup table of users and their
passwords; any user can connect using any username.  They will be accepted, as
long as not other user is currently logged on using the same username. The
server is designed to be able to have such methods and data structures
implemented, but this was never implemented as it wasn't required in the
project specifications and time ran out. 
There is currently no method for sending delivery reports.

\paragraph{Client}
Currently all messages are shown in the same message area;  details are given
regarding the messages that are sent and received:  sent messages are shown,
with a label indicating whom it was sent to, received messages show whom they
are from. But send time and time of receipt are discarded. I would have liked to
have had separate message tabs for different conversations, but this would have
required significant expansion of the message protocol I made up, with time
constraints not allowing it. Finally, when logging out, and back in, a new
listening thread is started for the client, without killing the old one.  This
has no visible effect, but means repeated logging in and out will eventually
bloat the resource usage.  

\section{Extra Features}
\paragraph{}

A few interesting features was added to the gui, such as clearing the text areas
upon gaining focus.  Echo messages was a fun extra feature.  


\section{Algorithms and data structures}

\paragraph{Messages}
All messages and communication between server and client happen using a
dedicated object called a Packet.  It has an integer field
called 'code' that determine what it is for.  The meaning of the code is defined
in the class called Code.java in the package called packet.  Currently it allows for
sending a message, echoing back, broadcasting, logging in, logging off and
requesting a list of online users. The codes are used by the server to correctly
determine where to process and send the packet and by the client to know whether
to display the data carried by the packet or how to react to it. A packet is
converted into a byte array using the methods in Serialize.java that was found
on a 'stackoverflow'
blog\footnote{http://stackoverflow.com/questions/2836646/java-serializable-object-to-byte-array}.

\paragraph{Server}
The thread for sending messages has a ConcurrentLinkedQueue of Packets.  This
structure was chosen, as it is thread safe.  Packets are popped of the front and
sent to the required recipients.  Packets coming into the server, are processed
by the so called 'listening' thread and put onto the ConcurrentLinkedQueue, if
appropriate.  A class, called User.java is dedicated toward maintaining a list
of online users.  It has two HashMaps, both with usernames and SocketChannels of
online users.  The one uses usernames as key, the other uses SocketChannels as
key.  This allows for using SocketChannels to find out which user went offline
unexpectedly, and also using username for looking up the SocketChannel of a user
to whom a message must be sent. 

\section{Tests Performed}

\paragraph{}
Multiple instances of the client program was created and connected to the
server.  Messages were sent between individual clients, broadcast and echoed.
Furthermore, clients were closed by closing the GUI window, by stopping it from
the terminal that started it using <ctrl-c> and by forcefully closing the terminal
window.  The server remained stable with each action, and could figure out which
user went offline, remove them from the appropriate data structures and send an
updated user list to the users still online.
When the server goes offline, the clients show this as a message on the
user board.  This does require the user to log off and back on when the server is
back up.  

\end{document}
This is never printed

\newpage


\paragraph{Answer}
\begin{verbatim}
   1 void adding_column_values(int** A, int m, int n, int* x) {
   2     int i, j;
   3     for(i = 0; i < m; i++) {
   4         for(j = 0; j < n; j++)
   5             x[j] += A[i][j];
   6     }
   7 }
\end{verbatim}


